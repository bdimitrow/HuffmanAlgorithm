\hypertarget{index_First}{}\section{Project\textquotesingle{}s aim}\label{index_First}
The aim of this project is to create an application that supports compression and decompression of text files using the Huffman algorithm. When started a menu appears on the screen and the user should input a digit between 1 and 6 in order to choose an action to be performed. The main functionality of the project is to support C\+O\+M\+P\+R\+E\+SS function that takes a file, reads its content and transforms it into a binary string. Moreover when a file is being compressed the compression ratio is printed on the screen whereas the binary sequence including an extra information about the Huffman tree(used when decompressing) are saved in a file. Another functionality is the D\+E\+C\+O\+M\+P\+R\+E\+SS function that takes an already compressed file and creates a new file containing the original content. The next function is called debug. It takes an already compressed file and prints the binary sequence as decimal numbers. When pressing 4 the program terminates. With 5 a text file is taken and compressed. The difference is from option 1 is that the binary sequence is translated into decimal sequence. By pressing 6 the user is able to decompress such files.\hypertarget{index_Github}{}\subsection{Github}\label{index_Github}
\href{https://github.com/bdimitrow/HuffmanAlgorithm}{\tt https\+://github.\+com/bdimitrow/\+Huffman\+Algorithm}\hypertarget{index_Second}{}\section{Compression Ratio}\label{index_Second}
\hypertarget{index_Third}{}\section{Demo}\label{index_Third}
\hypertarget{index_one}{}\subsection{After running}\label{index_one}
\hypertarget{index_two}{}\subsection{Original file content}\label{index_two}
Having a file named \textquotesingle{}test\+Equal.\+txt\textquotesingle{} with the following content\+: \hypertarget{index_three}{}\subsection{Compressed file content}\label{index_three}
The file after being compressed looks like that\+: \hypertarget{index_four}{}\subsection{Decompressed file content}\label{index_four}
The compressed file look like that after the decommpression\+: \hypertarget{index_five}{}\subsection{Debug option output}\label{index_five}
This is the output\+: 